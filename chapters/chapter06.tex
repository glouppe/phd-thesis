\chapter{Variable importances}\label{ch:importances}

\begin{remark}{Outline}
This chapter studies variable importance measures as computed from forests of
randomized trees. In Section~\ref{sec:6:importances}, we first present how
random forests can be used to assess the importance of input variables.  We
then derive in Section~\ref{sec:6:theory} a characterization in asymptotic
conditions and show that variable importances derived from totally randomized trees
offer a three-level decomposition of the information jointly  contained in the
input variables about the output. In Section~\ref{sec:6:variable-relevance}, we
show that this  characterization only depends on the relevant variables and
then discuss these ideas in Section~\ref{sec:6:variants} in the context of
variants closer to the Random Forest algorithm. Finally, we illustrate these
results on an artifical problem in Section~\ref{sec:6:illustration}.
\end{remark}

An important task in many scientific fields is the prediction of  a response
variable based on a set of predictor variables. In many situations though, the
aim is not only to make the most accurate predictions of the response but also
to identify which predictor variables are the most important to make these
predictions, e.g. in order to understand the underlying process. Because of
their applicability to a wide range of problems and their capability to both
build accurate models and, at the same time, to provide variable importance
measures, random forests have become a major data analysis tool used with
success in various scientific areas.

Despite their extensive use in applied research, only a couple of works have
studied the theoretical properties and statistical mechanisms of these
algorithms. \citet{zhao:2000}, \citet{breiman:2004},
\citet{biau:2008,biau:2012}, \citet{meinshausen:2006} and \citet{lin:2006}
investigated simplified to very realistic variants of these algorithms and
proved  the consistency of those variants. Little is known however regarding
the variable importances computed by random forests, and -- as far as we know
-- the work of~\citet{ishwaran:2007} is indeed the only theoretical study of
tree-based variable importance measures. In this chapter, we aim at filling
this gap and present a theoretical  analysis of the Mean Decrease Impurity
importance derived from ensembles of randomized trees.

% A forest of trees is impenetrable
% as far as simple interpretations of its mechanims go~\citep{breiman:2001}.


\section{Variable importances}
\label{sec:6:importances}

% surrogates
% in ensembles
% permutation

\section{Theoretical study}
\label{sec:6:theory}


\section{Relevance of variables}
\label{sec:6:variable-relevance}


\section{Variable importances in random forest variants}
\label{sec:6:variants}


\section{Illustration}
\label{sec:6:illustration}

